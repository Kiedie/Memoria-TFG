\section{Teorema de Aproximación Universal}

\subsection{Antecedentes}
\begin{frame}
    \begin{itemize}
        \item Teorema de Hahn-Banach
        \item Teorema de Representación de Riesz para funcionales lineales en $C_0(X)$
    \end{itemize}

    \begin{corollary}
    Sea $X$ un espacio normado, $M$ un subespacio de $X$. Si $x_0 \not\in \overline{M}$, existe un $f \in X^*$ tal que $f(x) = 0$ para todo $ x \in M$, $f(x_0) = 1$ y $||f|| = \frac{1}{d}$ donde $d=dist(x_0,M)$
    \end{corollary}
    
\end{frame} 

\subsection{Resultado}
\begin{frame}
    \begin{overprint}
    \onslide<1>
        \begin{theorem}[Teorema de la Aproximación Universal - G.Cybenko]
           Sea $\sigma$ una función continua sigmoidal. Entonces la suma finita dada por 
               \begin{equation}
                   G(x) = \sum_{j=1}^N \alpha_j \sigma(x_j^T x + \theta_j)
               \end{equation}
           \noindent es densa en $C(I_n)$. En otras palabras, dada cualquier función $f \in C(I_n)$ y $\epsilon > 0$, existe una suma G(X) que viene dada por la expresión de arriba de tal manera que 
           \begin{equation}
               |G(x) - f(x)| < \epsilon \quad \forall x \in I_n
           \end{equation}
        \end{theorem}
        
        
    \onslide<2>
            \begin{theorem}[Teorema de la Aproximación Universal - G.Cybenko]
           Sea $\sigma$ una función continua sigmoidal. Entonces la suma finita dada por 
               \begin{equation}
                   G(x) = \sum_{j=1}^N \alpha_j \sigma(x_j^T x + \theta_j)
               \end{equation}
           \noindent es densa en $C(I_n)$. En otras palabras, dada cualquier función $f \in C(I_n)$ y $\epsilon > 0$, existe una suma G(X) que viene dada por la expresión de arriba de tal manera que 
           \begin{equation}
               |G(x) - f(x)| < \epsilon \quad \forall x \in I_n
           \end{equation}
        \end{theorem}
        
    
        \begin{itemize}
            \item Existe una versión para problemas de clasificación
            \item Existen versiones con otras funciones de activación (ReLu, $\tanh$,...)
        \end{itemize}
        
\end{overprint}
\end{frame}