\section{Conclusiones}


\subsection{Conclusiones}

\begin{frame}
\pause
\begin{itemize}
    \setlength\itemsep{1.5em}
    \item La teoría matemática avala el potencial del aprendizaje profundo bajo ciertos supuestos.
    \pause
    \item {\color{TurkishRose}\textit{Enfoque Tradicional}} vs Enfoque Moderno.
    \pause
    \item Resultados manifiestamente mejorables. Limitaciones:
        \begin{itemize}
            \item Modelos de la competición $\rightarrow$ \textit{Normal vs Otro}, conocimiento a priori...
            \item Modelos de la literatura $\rightarrow$ Poca transportabilidad...
            \item Modelos propios $\rightarrow$ Tendencia por la clase mayoritaria...
        \end{itemize}
\end{itemize}
\end{frame}

\subsection{Trabajos Futuros}

\begin{frame}
\pause
\begin{itemize}
    \setlength\itemsep{1.5em}
    \item Mejorar la calidad de las bases de datos (desbalanceo, más ejemplos...)
    \pause
    \item Mejorar los modelos de la competición
    \pause
    \item Refinar los modelos propios
    \pause
    \item Probar nuevas técnicas: ensemble, stacking, boosting...
    \pause
    \item Aumentar el número de clases a clasificar
\end{itemize}
\end{frame}

