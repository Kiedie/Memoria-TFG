% !TeX root = ../libro.tex
% !TeX encoding = utf8

\setchapterpreamble[c][0.90\linewidth]{%
	\sffamily
  
	\par\bigskip
}

\chapter{Objetivos}\label{ch:objetivos}

    Los objetivos planteados al inicio del trabajo son los siguientes:
    
    \begin{itemize}
        \item Matemáticas: Se busca entender el funcionamiento y potencial de las redes profundas, para lo cual se busca:
            \begin{enumerate}
                \item Estudiar los conceptos matemáticos en los que se fundamentan
                \item Analizar algunos resultados sobre las redes neuronales
            \end{enumerate}
            
        \item Informática: Se busca abordar un problema de clasificación de señales de ECG para lo cual se busca:
            \begin{enumerate}
                \item Elaboración de modelos de aprendizaje profundo basados en redes neuronales recurrentes para la clasificación de arritmias.
                \item Estudiar y reproducir algunos modelos destacables de la literatura y realizar una comparación con los ya propuestos.
            \end{enumerate}
    \end{itemize}

    Los objetivos de la parte de matemáticas han sido cubiertos:
    \begin{enumerate}
        \item El primer objetivo queda cubierto por los capítulos \ref{ch:FundamentosMatematicos}, \ref{ch:DesigualdadHoeffding} y \ref{ch:AprendizajeEstadistico} en los que se estudia la teoría del aprendizaje estadístico que es en la que se fundamenta el aprendizaje automático, así como el aprendizaje profundo.
        \item El segundo objetivo queda cubierto por el capítulo \ref{ch:TAUniversal} en el que se presenta un importante teorema sobre la salida de las redes neuronales.
    \end{enumerate}
    
    Los objetivos de la parte de informática también han sido cubiertos:
    \begin{enumerate}
        \item El primer objetivo ha sido cubierto por los capítulos \ref{ch:descripcion_problema}, \ref{ch:contexto_experimental}, \ref{ch:desarrollo} y \ref{ch:analisis}.
        \item El segundo objetivo queda cubierto por los capítulos \ref{ch:desarrollo} y \ref{ch:analisis}.
    \end{enumerate}





    


\endinput