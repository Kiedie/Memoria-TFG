% DEFINICIÓN DE COMANDOS Y ENTORNOS

% CONJUNTOS DE NÚMEROS

  \newcommand{\N}{\mathbb{N}}     % Naturales
  \newcommand{\R}{\mathbb{R}}     % Reales
  \newcommand{\Z}{\mathbb{Z}}     % Enteros
  \newcommand{\Q}{\mathbb{Q}}     % Racionales
  \newcommand{\C}{\mathbb{C}}     % Complejos
  
  \newcommand{\E}{\mathbb{E}}     % Esperanza
  \newcommand{\e}{\epsilon}       % Épsilon
  \newcommand{\de}{\delta}        % Delta
  \newcommand{\Pb}{\mathbb{P}}    % P Probabilidad
  
  \newcommand{\X}{\mathcal{X}}    % X caligráfica
  \newcommand{\Y}{\mathcal{Y}}    % Y caligráfica
  \newcommand{\D}{\mathcal{D}}    % D caligráfica
  \newcommand{\A}{\mathcal{A}}    % A caligráfica
  \newcommand{\Pc}{\mathcal{P}}   % P caligráfica
  \newcommand{\Hc}{\mathcal{H}}   % H caligráfica
  \newcommand{\Rh}{\hat{R}}       % R gorro
  
  
  

% TEOREMAS Y ENTORNOS ASOCIADOS

  % \newtheorem{theorem}{Theorem}[chapter]
  \newtheorem*{teorema*}{Teorema}
  \newtheorem{teorema}{Teorema}[chapter]
  \newtheorem{proposicion}{Proposición}[chapter]
  \newtheorem{lema}{Lema}[chapter]
  \newtheorem{corolario}{Corolario}[chapter]

    \theoremstyle{definition}
  \newtheorem{definicion}{Definición}[chapter]
  \newtheorem{ejemplo}{Ejemplo}[chapter]

    \theoremstyle{remark}
  \newtheorem{observacion}{Observación}[chapter]
